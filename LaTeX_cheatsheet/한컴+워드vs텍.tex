\documentclass[11pt]{article}
\usepackage[left=25mm,right=25mm,top=30mm,bottom=30mm]{geometry}
\usepackage{amsmath} % math
\usepackage{amssymb} % math
\usepackage{graphicx} % to use \includegraphics{}
\usepackage{diagbox} % to make tables
\usepackage{multirow}
\usepackage{caption}
\usepackage{subcaption}
\usepackage{kotex}
\usepackage{color}
\usepackage[hidelinks]{hyperref}
\usepackage[per-mode=symbol]{siunitx}
\sisetup{inter-unit-product =$\cdot$} % http://tex.stackexchange.com/questions/59032/how-to-format-si-units
%\graphicspath{{images/}}
\begin{document}
	\begin{center}
		\Large 일반 워드프로세서\footnote{WYSIWYG(What You See Is What You Get)}와 \LaTeX correlation chart - 표, 그림 제외
	\end{center}
	\begin{flushright}
		by 14041 박승원 \\
		\today
	\end{flushright}
\begin{table}[h]
	\centering
	\begin{tabular}{|c|l|}
		\hline
		영어 & 영어 \\
		\hline
		한글 & \verb+\+usepackage[hangul]\{kotex\} \\
		\hline
		숫자 & 숫자 \\
		\hline
		italic & \verb+\+textit\{...\} 또는 \{\verb+\+it ... \} \\
		\hline
		bold & \verb+\+textbf\{...\} 또는 \{\verb+\+bf ...\} \\
		\hline
		취소선 & \verb+\+sout\{...\} \\
		\hline
		밑줄 & \verb+\+underline\{...\} \\
		\hline
		엔터 & 두번엔터 혹은 \verb+\+\verb+\+ 혹은 \verb+\+newline \\
		\hline
		따옴표 & `(키보드에서 Esc 밑에 있음), ' \\
		\hline
		\%, \&, \_, \{, \} & 각각 \verb+\+\%, \verb+\+\&, \verb+\+\_, \verb+\+\{, \verb+\+\} \\
		\hline
		\~{} & 쓰지 말자. \\
		\hline
		좌측/우측정렬 & \verb+\+begin\{flushleft\} ... \verb+\+end\{flushleft\} 또는 \{\verb+\+flushleft ...\}. 우측의 경우 flushright \\
		\hline
		중앙정렬 & \verb+\+begin\{center\} ... \verb+\+end\{center\} \\
		\hline
		폰트크기 &  \\
		\hline
		폰트 & (고급) \\
		\hline
		폰트 색 & \verb+\+usepackage\{color\} \\
		\hline
		각주 & \verb+\+footnote\{...\} \\
		\hline
		용지크기 & a4paper 등등 \\
		\hline
		편집용지 & (고급) \\
		\hline
		머리말/꼬리말 & \verb+\+usepackage\{fancyhdr\} \\
		\hline
		페이지나눔 & ... \\
		\hline
		줄간격조정 & ... \\
		\hline
		들여쓰기 조정 & \verb+\+usepackage\{indentfirst\} 등 \\
		\hline
	\end{tabular}
	\caption{asdf}
	\label{asdf}
\end{table}
{\tiny \verb+\+tiny} {\scriptsize \verb+\+scriptsize} {\footnotesize \verb+\+footnotesize} {\small \verb+\+small} {\normalsize \verb+\+normalsize} {\large \verb+\+large} {\Large \verb+\+Large} {\LARGE \verb+\+LARGE} {\huge \verb+\+huge} {\Huge \verb+\+Huge}
\end{document}
