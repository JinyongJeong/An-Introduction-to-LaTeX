\documentclass[11pt]{article}
\usepackage[left=25mm,right=25mm,top=30mm,bottom=30mm]{geometry}
\usepackage{amsmath} % math
\usepackage{amssymb} % math
\usepackage{graphicx} % to use \includegraphics{}
\usepackage{diagbox} % to make tables
\usepackage{multirow}
\usepackage{caption}
\usepackage{subcaption}
\usepackage{kotex}
\usepackage{color}
\usepackage[hidelinks]{hyperref}
\usepackage[per-mode=symbol]{siunitx}
\sisetup{inter-unit-product =$\cdot$} % http://tex.stackexchange.com/questions/59032/how-to-format-si-units
%\graphicspath{{images/}}

\begin{document}
\begin{center}
	\Large 동아리명 : 택도없는소리(Sound Without \TeX)
\end{center}
\begin{flushright}
	동아리원 : \\
	32기 : 박승원(기장), 이상헌(14080, 부기장), 이주찬(부기장), 김우찬, 신동윤(14056), 하석민,  \\
	33기 : 김경태 \\
	지도교사 : 목진욱 선생님(예정)
\end{flushright}
\section{활동 내용 및 방법}
\begin{enumerate}
\item 간단한 sample document 및 \LaTeX 책자 제작
\item R\&E, 졸업논문, 휴먼테크논문대회, 일반화학실험 보고서 등등 cls 및 tex 파일 {\bf 제작, 수정, 배포}
\item Fundamentals of \LaTeX 강의
\begin{enumerate}
	\item 이수 이후 자기주도적인 \LaTeX 연습 및 학습이 가능하게끔 한다.
	\item koTeXLive, TeXstudio 설치
	\item \LaTeX 사용 목적, 편리함 및 보편성
	\item \LaTeX 기본 구조(wikipedia에의 비유) - 책자 제작이 시급함!!
	\item \checkmark 3학년 학생이 1학년 기초 R\&E 전체교육 도움/실시
\end{enumerate}
\end{enumerate}

\section{threats}
\begin{itemize}
\item 저작권(cls 파일, example 파일) - 동아리 가입시 CC 동의 의무화?
\item 어려울 것이라는 막연한 공포의 확산
\item 동아리의 지속 가능성
\end{itemize}

\end{document}
