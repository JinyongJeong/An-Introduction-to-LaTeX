\documentclass[12pt]{beamer}
\usetheme{Hannover}
\usepackage{graphicx}
\usepackage{booktabs}
\usepackage[english]{babel}
\usepackage{kotex}
%\usepackage[pdfencoding=auto]{hyperref}
\hypersetup{pdfencoding=auto}
\usepackage{ulem}
\usepackage[per-mode=symbol]{siunitx}
\sisetup{inter-unit-product =$\cdot$}
\usepackage{verbatim}
\setbeamertemplate{caption}[numbered]
\graphicspath{{images/}}

\title[\LaTeX - Day 3]{\LaTeX 입문 - Day 3}

\author{경기과학고 \TeX 사용자협회}
\institute[GSHSTeXSociety]{\url{latex.gs.hs.kr}}
\date{마지막 수정일 : \today}

%\setbeamertemplate{navigation symbols}{}%to suppress navigation tools

\AtBeginSection[]{
	\begin{frame}
		\vfill
		\centering
		\begin{beamercolorbox}[sep=8pt,center,shadow=true,rounded=true]{title}
			\usebeamerfont{title}\insertsectionhead\par%
		\end{beamercolorbox}
		\vfill
	\end{frame}
}
\begin{document}

\begin{frame}
\titlepage % Print the title page as the first slide
\end{frame}

\begin{frame}{지난 시간에는}
	\begin{itemize}
		\item 라벨링 및 상호 참조
		\item 그림과 표의 삽입
		\item 논문 작성에 도움이 되는 팁들
	\end{itemize}
	이번 시간에는 tikz를 사용하여 그림을 그리는 방법에 대해 알아보겠다.
\end{frame}

\section{점}
\begin{frame}{\secname}

\end{frame}

\begin{frame}{\secname}

\end{frame}


\section{선}

\subsection{직선}
\subsection{화살표}
\subsection{곡선}

\section{면}
\subsection{다각형}
\subsection{원}

\section{좌표 평면}
\subsection{좌표 계}
\subsection{함수}

\section{Extra Tips}


\end{document}