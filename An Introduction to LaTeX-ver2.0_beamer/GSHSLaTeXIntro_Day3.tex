\documentclass[12pt]{beamer}
\usetheme{Hannover}
\usepackage{graphicx}
\usepackage{booktabs}
\usepackage[english]{babel}
\usepackage{kotex}
%\usepackage[pdfencoding=auto]{hyperref}
\hypersetup{pdfencoding=auto}
\usepackage{ulem}
\usepackage[per-mode=symbol]{siunitx}
\sisetup{inter-unit-product =$\cdot$}
%\usepackage{verbatim}
\usepackage{listings}
\lstset{language = [LaTeX]{TeX},basicstyle=\footnotesize\linespread{1}\ttfamily,tabsize=4,backgroundcolor=\color[gray]{0.9},showstringspaces=false}
\usepackage{tikz}
\setbeamertemplate{caption}[numbered]
\graphicspath{{images/}}

\title[\LaTeX - Day 3]{\LaTeX 입문 - Day 3}

\author{경기과학고 \TeX 사용자협회}
\institute[GSHSTeXSociety]{\url{latex.gs.hs.kr}}
\date{마지막 수정일 : \today}

%\setbeamertemplate{navigation symbols}{}%to suppress navigation tools

\AtBeginSection[]{
	\begin{frame}
		\vfill
		\centering
		\begin{beamercolorbox}[sep=8pt,center,shadow=true,rounded=true]{title}
			\usebeamerfont{title}\insertsectionhead\par%
		\end{beamercolorbox}
		\vfill
	\end{frame}
}
\begin{document}

\begin{frame}
\titlepage % Print the title page as the first slide
\end{frame}

\begin{frame}{지난 시간에는}
	\begin{itemize}
		\item 라벨링 및 상호 참조
		\item 그림과 표의 삽입
		\item 논문 작성에 도움이 되는 팁들
	\end{itemize}
	\vfill
	이번 시간에는 tikz를 사용하여 그림을 그리는 방법에 대해 알아보겠다.
\end{frame}

\section{들어가기 전에..}
\begin{frame}[fragile]{패키지}
	tikz를 사용하기 위해서는
	\begin{lstlisting}
\usepackage{tikz}
	\end{lstlisting}
	로 tikz 패키지를 추가해야 한다
	\vfill
	\begin{lstlisting}
\usetikzlibrary{}
	\end{lstlisting}
	로 라이브러리를 추가할 수도 있다
\end{frame}

\begin{frame}[fragile]{사용 환경}
	tikz로 그림을 그리는 방법은 다음 두 가지가 있다
	\vfill
	\begin{lstlisting}
\begin{tikzpicture}[option]
	(tikz command)
\end{tikzpicture}
	\end{lstlisting}
	\vfill
	\begin{lstlisting}
\tikz[option]{(tikz command)}
	\end{lstlisting}
	\vfill
	tikz 명령어 맨 뒤에는 반드시 세미콜론(;)을 붙여야 한다
\end{frame}

\begin{frame}{위치}
	tikz는 데카르트 좌표계를 사용한다
	\vfill
	오른쪽으로 가면 x좌표가 증가하고

	위쪽으로 가면 y좌표가 증가한다
	\vfill
	화면의 중심은 (0,0)이 아니고 내가 그린 그림이 알맞은 위치에 오도록 알아서 맞춰진다
\end{frame}

\section{점}
\begin{frame}{\secname}{표기}
	점은 (x,y)로 나타낸다
	
	x와 y는 각각 x좌표와 y좌표이다
	\vfill
	(10cm,2pt)같이 단위를 쓸 수 있는데 단위를 쓰지 않으면 기본으로 cm가 적용된다
\end{frame}

\begin{frame}[fragile]{\secname}{node}
	다음과 같이 좌표를 지정할 수 있다
	\begin{lstlisting}
\node[option](name) at (coordinate){text};
\coordinate[option](name) at (coordinate);
	\end{lstlisting}
	\vfill
	node는 점의 위치에 글을 쓸 수 있고 
	
	coordinate는 못한다
	\vfill
	점은 화면에 나타나지 않는다\\
	나타나는 것은 node로 쓴 글 뿐이다
\end{frame}

\begin{frame}[fragile]{\secname}{path}
	path를 사용해 여러개의 점을 한꺼번에 지정할 수 있다
	
	\begin{lstlisting}
\path (coordinate) node[options](name){text}
(coordinate) node [options](name){text} ... ;
	\end{lstlisting}
	
	원하는 만큼 node로 점을 지정하면 된다
	\vfill
	좌표를 먼저 쓰는 것에 주의하라
	
	뒤에서는 점의 좌표를 먼저 쓰는 형태가 흔하다

\end{frame}

\begin{frame}[fragile]{\secname}{example}
	\begin{lstlisting}
\node (a) at (10,0){This is a};
\coordinate (b) at (0,10);
\path (0,0) node(x) {X}
(10,10) node(y) {Y};
	\end{lstlisting}
	
\begin{center}
	\begin{tikzpicture}
	\node (a) at (3,0){This is A};
	\coordinate (b) at (0,3);
	\path (0,0) node(o) {O}
		  (3,3) node(c) {C};
	\end{tikzpicture}
\end{center}

\end{frame}


\section{선}

\subsection{직선}
\subsection{화살표}
\subsection{곡선}

\section{면}
\subsection{다각형}
\subsection{원}

\section{좌표 평면}
\subsection{좌표 계}
\subsection{함수}

\section{Extra Tips}


\end{document}