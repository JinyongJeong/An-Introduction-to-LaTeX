%\documentclass[a4paper,12pt]{article}
\documentclass[12pt]{standalone}
\usepackage{tikz}
\usepackage{verbatim} % to use \begin{comment} ... \end{comment}
\usetikzlibrary{shapes,arrows,positioning,external}
\begin{document}
	
	% standalone를 사용하면 pdf 문서가 마치 하나의 이미지 파일처럼, margin 을 포함하지 않고 내용물만을 포함하게 됩니다.
	% 아래 예시에는 6개의 블록 다이어그램이 있습니다. 필요한 부분을 제외하고 comment 처리하여 pdf 로 만들면 이미지처럼 사용할 수 있습니다.
	
	
	\tikzstyle{block} = [draw, fill=blue!20, rectangle, 
	minimum height=2em, minimum width=3em]
	\tikzstyle{block-blue} = [draw, fill=blue!30, rectangle, 
	minimum height=2em, minimum width=3em]
	\tikzstyle{block-red} = [draw, fill=red!40, rectangle, 
	minimum height=2em, minimum width=3em]
	\tikzstyle{block-green} = [draw, fill=green!40, rectangle, 
	minimum height=2em, minimum width=3em]
	\tikzstyle{block-orange} = [draw, fill=orange!40, rectangle, 
	minimum height=2em, minimum width=3em]
	\tikzstyle{sum} = [draw, fill=blue!20, circle, node distance = 1cm]
	\tikzstyle{input} = [draw, coordinate]
	\tikzstyle{output} = [draw, coordinate]
	\tikzstyle{pinstyle} = [pin edge={to-,thin,black}]
	
	
	
	\begin{comment}
	\begin{tikzpicture}[auto, node distance=2cm,>=latex'] % Force Control - Constant Power
	
	%\node[input, name=input] {};
	\node (input) {$F_{ref}$};
	\node[block-blue, right of=input](KFP) {$K_{F,P}$};
	\node[block-red, right of=KFP](calc) {$\frac{c}{C_{th}s+\lambda}$};
	\node at (6,0) (output) {$F$};
	% \node[output, right of = calc] (output) {$F$};
	
	\draw [draw,->] (input) -- node {} (KFP);
	\draw [->] (KFP) -- node {$P$} (calc);
	\draw [->] (calc) -- node {} (output);
	
	\end{tikzpicture}
	
	\begin{tikzpicture}[auto, node distance=2cm,>=latex'] % Force Control - Open-loop Control
	\node (input) {$F_{ref}$};
	\node[block-orange, right of=input] (tau-s) {$\frac{\tau s+1}{\tau_{des} s+1} $};
	\node[block-blue, right of=tau-s] (KFP) {$K_{F,P}$};
	\node[block-red, right of=KFP] (calc) {$\frac{c}{C_{th}s+\lambda}$};
	\node at (8,0) (output) {$F$};
	
	\draw [draw,->] (input) -- node {} (tau-s);
	\draw [->] (tau-s) -- node {$u$} (KFP);
	\draw [->] (KFP) -- node {$P$} (calc);
	\draw [->] (calc) -- node {} (output);
	\end{tikzpicture}
	\begin{tikzpicture}[auto, node distance=2cm,>=latex'] % Force Control - Feedback Control
	\node(input) {$F_{ref}$};
	\coordinate[right = 0.2cm of input] (input-sum1) {};
	\node[sum, right of=input-sum1, node distance = 0.5cm] (sum1) {};
	\node[block-green, right of=sum1, node distance=1.5cm] (Kp) {$K_{p}$};
	\node[sum, right of=Kp, node distance = 1.3cm] (sum2) {};
	\node[block-orange, right of=sum2, node distance=1.3cm] (tau-s) {$\frac{\tau s+1}{\tau_{des} s+1} $};
	\node[block-blue, right of=tau-s] (KFP) {$K_{F,P}$};
	\node[block-red, right of=KFP] (calc) {$\frac{c}{C_{th}s+\lambda}$};
	%\node[below of=KFP] (middle) {};
	%\coordinate[below of=KFP] (middle) {};
	\coordinate[below = 0.2cm of KFP] (middle) {};
	\coordinate[above = 0.2cm of Kp] (middle2) {};
	\node at (11,0) (output) {$F$};
	
	\draw [draw,->] (input) -- node {} (sum1);
	\draw [->] (sum1) -- node {$e$} (Kp);
	\draw [->] (Kp) -- node {} (sum2);
	\draw [->] (sum2) -- node {} (tau-s);
	\draw [->] (tau-s) -- node {$u$} (KFP);
	\draw [->] (KFP) -- node {$P$} (calc);
	\draw [->] (calc) -- node [name=QQ] {}(output);
	%\draw [->] (QQ) |- (middle) -| (sum1); 
	\draw (QQ) |- (middle);
	\draw [->] (middle) -| node[pos = 0.85] {\tiny{$-$}} node[left,pos=1.45] {\tiny{$+$}} (sum1);
	\draw (input-sum1) |- (middle2);
	\draw [->] (middle2) -| node[left,pos=0.85]{\tiny{$+$}} node[left,pos=1.45]{\tiny{$+$}} (sum2);
	\end{tikzpicture}
	\begin{tikzpicture}[auto, node distance=2cm,>=latex'] % Position Control - Constant Power
	
	%\node[input, name=input] {};
	\node (input) {$\theta_{ref}$};
	\node[block-blue, right of=input](KFP) {$K_{\theta,P}$};
	\node[block-red, right of=KFP](calc) {$\frac{c/2k}{C_{th}s+\lambda}$};
	\node at (6,0) (output) {$\theta$};
	% \node[output, right of = calc] (output) {$F$};
	
	\draw [draw,->] (input) -- node {} (KFP);
	\draw [->] (KFP) -- node {$P$} (calc);
	\draw [->] (calc) -- node {} (output);
	
	\end{tikzpicture}
	\begin{tikzpicture}[auto, node distance=2cm,>=latex'] % Position Control - Open-loop Control
	\node (input) {$\theta_{ref}$};
	\node[block-orange, right of=input] (tau-s) {$\frac{\tau s+1}{\tau_{des} s+1} $};
	\node[block-blue, right of=tau-s] (KFP) {$K_{\theta,P}$};
	\node[block-red, right of=KFP] (calc) {$\frac{c/2k}{C_{th}s+\lambda}$};
	\node at (8,0) (output) {$\theta$};
	
	\draw [draw,->] (input) -- node {} (tau-s);
	\draw [->] (tau-s) -- node {$u$} (KFP);
	\draw [->] (KFP) -- node {$P$} (calc);
	\draw [->] (calc) -- node {} (output);
	\end{tikzpicture}
	\end{comment}
	\begin{tikzpicture}[auto, node distance=2cm,>=latex'] % Position Control - Feedback Control
	\node(input) {$\theta_{ref}$};
	\coordinate[right = 0.2cm of input] (input-sum1) {};
	\node[sum, right of=input-sum1, node distance = 0.5cm] (sum1) {};
	\node[block-green, right of=sum1, node distance=1.5cm] (Kp) {$K_{p}$};
	\node[sum, right of=Kp, node distance = 1.3cm] (sum2) {};
	\node[block-orange, right of=sum2, node distance=1.3cm] (tau-s) {$\frac{\tau s+1}{\tau_{des} s+1} $};
	\node[block-blue, right of=tau-s] (KFP) {$K_{\theta,P}$};
	\node[block-red, right of=KFP] (calc) {$\frac{c/2k}{C_{th}s+\lambda}$};
	%\node[below of=KFP] (middle) {};
	%\coordinate[below of=KFP] (middle) {};
	\coordinate[below = 0.2cm of KFP] (middle) {};
	\coordinate[above = 0.2cm of Kp] (middle2) {};
	\node at (11,0) (output) {$\theta$};
	
	\draw [draw,->] (input) -- node {} (sum1);
	\draw [->] (sum1) -- node {$e$} (Kp);
	\draw [->] (Kp) -- node {} (sum2);
	\draw [->] (sum2) -- node {} (tau-s);
	\draw [->] (tau-s) -- node {$u$} (KFP);
	\draw [->] (KFP) -- node {$P$} (calc);
	\draw [->] (calc) -- node [name=QQ] {}(output);
	%\draw [->] (QQ) |- (middle) -| (sum1); 
	\draw (QQ) |- (middle);
	\draw [->] (middle) -| node[pos = 0.85] {\tiny{$-$}} node[left,pos=1.45] {\tiny{$+$}} (sum1);
	\draw (input-sum1) |- (middle2);
	\draw [->] (middle2) -| node[left,pos=0.85]{\tiny{$+$}} node[left,pos=1.45]{\tiny{$+$}} (sum2);
	
	
	\end{tikzpicture}
\end{document}